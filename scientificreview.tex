%%%%%%%%%%%%% Document Type %%%%%%%%%%%%%
\documentclass[twoside,twocolumn]{article}                          % Article with 2 column
\usepackage[english]{babel}                                         % English document
\usepackage{enumitem}                                               % Items enumeration
\usepackage{array}                                                  % Create tables
\usepackage{eurosym}                                                % € symbol
\usepackage{hyperref}                                               % Internet and pdf links
\usepackage{color}                                                  % Colors for text
\usepackage{fancyhdr}                                               % Type of LaTeX decoration
\usepackage{titlesec}                                               % Style selection
\usepackage{lscape}                                                 % Help top create a large document
\usepackage{amssymb}                                                % Maths symbols
\usepackage[margin=0.8in]{geometry}                                 % Margin
\usepackage{blindtext}                                              % To create blank zones
\usepackage{microtype}                                              % Micro texte
\usepackage[hang, small,labelfont=bf,up,textfont=it,up]{caption}    % Caption modification
\usepackage{lettrine}                                               % First enlarged letter
\usepackage{titlesec}                                               % Allows customization of titles
\usepackage{titling}                                                % Allows customization of title too
\usepackage{authblk}                                                % For multiple authors
\usepackage{graphicx}                                               % Use of graphics and images
\usepackage{etoolbox}                                               % Remove the "References" title of
\patchcmd{\thebibliography}{\section*{\refname}}{}{}{}              % the bibliography


%%%%%%%%%%%%% Configurations %%%%%%%%%%%%%
\linespread{1.05}
\setlist[itemize]{noitemsep}                                        % Make itemize lists more compact
\setlength{\abovecaptionskip}{0pt plus 0pt minus 0pt}               % Change length above caption
\pagestyle{fancy}                                                   % LaTeX style
\fancyhead{}                                                        % Blank out the default header
\fancyfoot{}                                                        % Blank out the default footer
\fancyhead[C]{Improvement of swimming by dint of biomimicry - December 2020 - Vol. I, No. 1}% Custom header text

%%%%%%%%%%%%% Title informations %%%%%%%%%%%%%
\setlength{\droptitle}{-4\baselineskip}                             % Move the title up
\pretitle{\begin{center}\Huge\bfseries}                             % Article title formatting
\posttitle{\end{center}}                                            % Article title closing formatting
\title{Improvement of swimming by dint of biomimicry}               % Title
\author[3]{LEFAY Paul}                                              % Authors
\author[1]{CESBRON Théo}
\author[2]{HERVE Alexis}
\author[2]{GELINEAU Emmanuel}
\author[2]{GUYON Soren}
\affil[1]{BIO}                                                     % Affiliations
\affil[2]{EOC}
\affil[3]{CSS}

\renewcommand{\maketitlehookd}{%
\begin{abstract}
  Swimsuits play a role in swimmer’s performance as it can be visible during 
  national, world or Olympic events. It creates the small gap necessary to make 
  a difference between competitors. When someone is swimming, he must overcome 
  hydrodynamic resistance. This scientific article describes both aerodynamics 
  and hydrodynamics comparative evaluation. First, we compare two commercially 
  available swimsuits to know which one the best is to overcome hydrodynamics. 
  Second, 9 swimmers (5 males and 4 females) wearing the same Fastskin TM 
  swimsuits performed a prone streamlined glide and maximum effort flutter kick 
  at 1.6, 2.2 and 2.8 m/s. Before that, they were weighed in a hydrostatic tank 
  and then towed via a mechanical winch on the surface and 0.4 m deep at 1.6, 2.2 
  and 2.8 m/s each towing velocity and depth. \\

  \textbf{Keywords} : Swimsuit, aerodynamics, hydrodynamic, drag, performance. \\
\end{abstract}
}
\setlength{\columnsep}{25px}                                        % Change the space between 2 columns

%%%%%%%%%%%%% Start of the document %%%%%%%%%%%%%
\begin{document}

%%%%%%%%%%%%% Configurations %%%%%%%%%%%%%
\renewcommand\thesection{\Roman{section}}                           % Roman numerals for the sections
\renewcommand\thesubsection{\roman{subsection}}                     % Roman numerals for subsections

%%%%%%%%%%%%% Header and footer %%%%%%%%%%%%%

\cfoot{\thepage}

\maketitle{}										                                    %Génération du titre

\newpage
\section{Introduction}
As astronomy or transport, sport is a sector where innovations are numerous. For a longtime, sportsmen and sportswomen were trying to achieve better performances by training a lot. The fact is that they finally arrived to the limits of performances. A way of pushing those limits is to study new ways of training and additional things to help perform. One of the most known examples is the fastskin. The fastskin is a swimsuit that enables swimmers to go faster. It is based on the same principle of shark skin.
This kind of swimsuit is the perfect picture of biomimicry. Biomimicry is the idea of observing wildlife and transposing its concept in life to solve problems or improve our way of doing things. This process has enabled humans to create the velcro, the result of observing a plant that can hang on many surfaces.


\section{Method}
\subsection{Experiment 1}
In order to obtain aerodynamic properties experimentally, a 110 mm diameter PVC cylinder was made. He was vertically supported on a six component force sensor. The aerodynamic forces were measured in a wind tunnel air speed from 10 km/h to 120 km/h with a 10 km/h step. Two full-body swimsuit materials have been selected for this study as they are officially used in the world and Olympic events.
Speedo Fast Skin-II (FS-II) : 75\% Polyamide and 25\% Elastane 
Speedo LZR Racer :  70\% Polyamide (polyurethane) and 30\% Elastane 
First we tested the bare cylinder, then we wrapped them with the swimsuit. The first swimsuit was tested at several fibre orientations :
\begin{enumerate}
  \item V-shape
  \item Seam position
  \item $\wedge$-shape
\end{enumerate}

The second swimsuit was tested as a seam position only because it does not have V-shape. 

\subsection{Experiment 2}
9 swimmers (5 males and 4 females) wearing the same Fastskin TM swimsuits performed a prone streamlined glide and maximum effort flutter kick at 1.6, 2.2 and 2.8 m/s. Before that, they were weighed in a hydrostatic tank and then towed via a mechanical winch on the surface and 0.4 m deep at 1.6, 2.2 and 2.8 m/s each towing velocity and depth. 
Hydrostatic weight was recorded while wearing standard swimsuits and full-length Fastskin TM swimsuits to determine whether either suit provided a buoyancy benefit. The hydrostatic weight was measured to the nearest 0.098 N while the subject was submerged in a stationary position following a maximum exhalation. The procedure was repeated randomly in sets of five with the participant wearing the full-length Fastskin TM swimsuit and a standard swimsuit. The average of the last three readings was recorded. The towing system consisted of a steel wire attached to a variable control motorised winch. The towing velocity was adjustable in increments of 0.1 m/s. Depth was controlled via an adjustable pulley system fixed on to the pool wall which allowed the towing force to be applied horizontally at the required depth. 


% \begin{itemize}
%   \item Donec dolor arcu, rutrum id molestie in, viverra sed diam
%   \item Curabitur feugiat
%   \item turpis sed auctor facilisis
%   \item arcu eros accumsan lorem, at posuere mi diam sit amet tortor
%   \item Fusce fermentum, mi sit amet euismod rutrum
% \end{itemize}

\section{Results}
\subsection{First}
The following conclusions are drawn, based on the experimental work presented here: • The average drag coefficient of FS-II material at high speeds (over 70 km/h) is approximately 0.62 and the average drag coefficient for LZR material (without polyurethane) at high speeds (over 90 km/h) is approximately 0.56. • The fibre orientation has significant effect on aerodynamic drag and the optimal orientation can reduce the hydrodynamic drag. • The seam weld of LZR has minimum or no effect on aerodynamic drag • The polyurethane material and the bare cylinder display the similar aerodynamic drag. 

\section{Discussion}
Those results mean that between a classic swimsuit and the fastskin, there is an increase of performances between 5 to 10 percent. This example shows that by studying nature and animals, we can improve our skills and our capabilities. Such an improvement enables athletes to compete at a higher level and as we saw during the 2004 Olympics where a lot of time record were broken.
This suit comes as an additional part for the human. Searchers can go further in the biomimicry by looking for new improvements for the human body, which can be used in sport but also to prevent some damages for example. 


\section{Conclusion}
The initial problem was the fact that we arrive at a point where performances reach his limit. With the search in the fastskin, results have been improved by 5 to 10 percent. It proves that even if the human body can reach a limit, by working on the nature, we can improve our capabilities to reach new records.

\cite{Swimsuits}
\cite{Fastskin}

\begin{figure}[!h]
  \begin{center}
    \includegraphics[scale=0.2]{eseo.JPG}
  \end{center}
  \caption{Logo ESEO}
\end{figure}


\section{Acknowledgements}
We want to thanks N Benjanuvatra; G Dawson; B.A. Blanksby; B.C. Elliott for their article “Comparison of buoyancy, passive and net active drag forces between Fastskin™ and standard swimsuits”. It enables us to understand how the fastskin was tested in a hydrodynamic way. 
To complete, we are grateful to Hazim Moria; Harun Chowdhury; Firoz Alam; Aleksandar Subic; Alexander John Smits; Rahim Jassim; Nasser Suliman Bajaba for their article “Contribution of swimsuits to swimmer’s performance”. It helps us to understand how the fastskin was tested in an aerodynamic way. 


\section{References}
\bibliographystyle{plain}
\bibliography{biblio}

\end{document}